% Unofficial UofT Poster template.
% A fork of the UMich template https://www.overleaf.com/latex/templates/university-of-michigan-umich-poster-template/xpnqzzxwbjzc
% which is fork of the MSU template https://www.overleaf.com/latex/templates/an-unofficial-poster-template-for-michigan-state-university/wnymbgpxnnwd
% which is a fork of https://www.overleaf.com/latex/templates/an-unofficial-poster-template-for-new-york-university/krgqtqmzdqhg
% which is a fork of https://github.com/anishathalye/gemini
% also refer to https://github.com/k4rtik/uchicago-poster

\documentclass[final]{beamer}

% ====================
% Packages
% ====================

\usepackage[T1]{fontenc}
\usepackage[utf8]{inputenc}
\usepackage{lmodern}
\usepackage[size=custom, width=122,height=91, scale=1.2]{beamerposter}

% Add the sty directory to the search path
\makeatletter
\def\input@path{{../sty/}}
\makeatother

\usetheme{gemini}
\usecolortheme{uoft}
\usepackage{graphicx}
\usepackage{booktabs}
\usepackage{tikz}
\usepackage{pgfplots}
\pgfplotsset{compat=1.14}
\usepackage{anyfontsize}

% ====================
% Lengths
% ====================

% If you have N columns, choose \sepwidth and \colwidth such that
% (N+1)*\sepwidth + N*\colwidth = \paperwidth
\newlength{\sepwidth}
\newlength{\colwidth}
\setlength{\sepwidth}{0.025\paperwidth}
\setlength{\colwidth}{0.3\paperwidth}

\newcommand{\separatorcolumn}{\begin{column}{\sepwidth}\end{column}}

% ====================
% Title
% ====================

\title{Tipler: Campo Eléctrico}

\author{@te\_vas\_a\_enterar}

\institute[shortinst]{Alberto de Santos}

% ====================
% Footer (optional)
% ====================

\footercontent{
  \href{https://github.com}{Github: https://github.com} \hfill
ICLR 2023 \hfill
  \href{mailto:youremail@cs.toronto.edu}{youremail@cs.toronto.edu}}
% (can be left out to remove footer)

% ====================
% Logo (optional)
% ====================

% use this to include logos on the left and/or right side of the header:
% Left: institution
% \logoright{\includegraphics[height=8cm]{logos/logo.png}}
% Right: funding agencies and other affilations 
%\logoright{\includegraphics[height=7cm]{logos/NSF.eps}}
% ====================
% Body
% ====================

\begin{document}



\begin{frame}[t]
\begin{columns}[t]
\separatorcolumn

\begin{column}{\colwidth}

  \begin{block}{Enunciado}
    Una carga puntual positiva de magnitud $+2.5 \mu C$ se encuentra en el centro de una corteza conductora esférica sin carga, 
        de radio interior $60cm$ y de radio exterior $90cm$.
    \begin{enumerate}[(a)]
        \item Determinar las densidades de carga sobre las superficies interior y exterior de la corteza y la $Q_T$ sobre cada superficie.
        \item Determinar el campo eléctrico en cualquier punto.
        \item Repetir (a) y (b) para el caso en que una carga neta de $+3.5 \mu C$ se sitúa sobre la corteza.
    \end{enumerate}

  \end{block}

  \begin{block}{Datos}

    Según el enunciado, contamos con los siguientes datos:

    \begin{itemize}
      \item \textbf{Carga} $q = +2.5 \mu C$
      \item \textbf{Radio interior} $r_{int} = 60cm$
      \item \textbf{Radio exterior} $r_{ext} = 90cm$
      \item \textbf{Conductor sin carga}, esto muy importante.
    \end{itemize}

  \end{block}

  \begin{alertblock}{Planteamiento}

    La \textbf{ley de Gauss} permite relacionar el flujo eléctrico que atraviesa una superficie cualquiera cerrada $S$ con la carga encerrada
    en dicha superficie:

    \begin{equation}
        \oint_S \vec{E} \cdot d\vec{S} = \frac{Q_{enc}}{\varepsilon_0}
        \label{eq:gauss}
    \end{equation}

    donde:

    \begin{itemize}
      \item $\vec{E}$ es el \textbf{campo eléctrico} que atraviesa la superficie $S$.
      \item la carga $Q_{enc}$ es la \textbf{carga eléctrica} encerrada por la superificie $S$.
      \item $\varepsilon_0$ es la permeabilidad del vacío.
    \end{itemize}

    Esta relación nos permitirá calcular el campo eléctrico en cualquier punto del espacio.

  \end{alertblock}

\end{column}

\separatorcolumn

\begin{column}{\colwidth}

  \begin{block}{Apartado (a)}

    Aplicando la distribución de cargas, y considerando el hecho de que se trata de una corteza conductora, 
    sabemos que:

    \begin{itemize}
      \item Sobre la superficie interior ($r_{int}$) aparece una \textbf{distribución superficial} de carga de valor $-q$.
      \begin{itemize}
        \item Es una \textbf{distribución superficial} de carga porque se trata de una superficie \textbf{conductora} y 
        la carga se distribuye siempre uniformemente por la superficie cuando se trata de un material conductor, nunca en el interior.
        \item Si fuera un material \textbf{aislante o no conductor}, la carga se distribuiría uniformemente por todo el volumen del material.
      \end{itemize}
      \item Sobre la superficie exterior ($r_{ext}$) aparece una distribución superficial de carga de valor $+q$, ya que la 
      carga total sobre el conductor es inicialmente \textbf{sin carga}.
        \begin{itemize}
            \item Si la carga inicial hubiese sido $q_{inicial} \neq 0$ (no descargada), 
            la carga sobre la superficie exterior sería $q_{inicial} = q_{ext} + q_{int}$.
            \item Siempre hay que aplicar los principios de conservación de la carga (electrostática).
        \end{itemize}
    \end{itemize}
    Por lo tanto:
    \begin{itemize}
        \item En la \textunderscore{superficie interna}:
    \end{itemize}

  \end{block}

  \begin{block}{Fusce aliquam magna velit}

    Et rutrum ex euismod vel. Pellentesque ultricies, velit in fermentum
    vestibulum, lectus nisi pretium nibh, sit amet aliquam lectus augue vel
    velit. Suspendisse rhoncus massa porttitor augue feugiat molestie. Sed
    molestie ut orci nec malesuada. Sed ultricies feugiat est fringilla
    posuere.

  \end{block}

  \begin{block}{Nam cursus consequat egestas}

    Nulla eget sem quam. Ut aliquam volutpat nisi vestibulum convallis. Nunc a
    lectus et eros facilisis hendrerit eu non urna. Interdum et malesuada fames
    ac ante \textit{ipsum primis} in faucibus. Etiam sit amet velit eget sem
    euismod tristique. Praesent enim erat, porta vel mattis sed, pharetra sed
    ipsum. Morbi commodo condimentum massa, \textit{tempus venenatis} massa
    hendrerit quis. Maecenas sed porta est. Praesent mollis interdum lectus,
    sit amet sollicitudin risus tincidunt non.

    Etiam sit amet tempus lorem, aliquet condimentum velit. Donec et nibh
    consequat, sagittis ex eget, dictum orci. Etiam quis semper ante. Ut eu
    mauris purus. Proin nec consectetur ligula. Mauris pretium molestie
    ullamcorper. Integer nisi neque, aliquet et odio non, sagittis porta justo.

    \begin{itemize}
      \item \textbf{Sed consequat} id ante vel efficitur. Praesent congue massa
        sed est scelerisque, elementum mollis augue iaculis.
        \begin{itemize}
          \item In sed est finibus, vulputate
            nunc gravida, pulvinar lorem. In maximus nunc dolor, sed auctor eros
            porttitor quis.
          \item Fusce ornare dignissim nisi. Nam sit amet risus vel lacus
            tempor tincidunt eu a arcu.
          \item Donec rhoncus vestibulum erat, quis aliquam leo
            gravida egestas.
        \end{itemize}
      \item \textbf{Sed luctus, elit sit amet} dictum maximus, diam dolor
        faucibus purus, sed lobortis justo erat id turpis.
      \item \textbf{Pellentesque facilisis dolor in leo} bibendum congue.
        Maecenas congue finibus justo, vitae eleifend urna facilisis at.
    \end{itemize}

  \end{block}

\end{column}

\separatorcolumn

\begin{column}{\colwidth}

  \begin{exampleblock}{A highlighted block containing some math}

    A different kind of highlighted block.

    $$
    \int_{-\infty}^{\infty} e^{-x^2}\,dx = \sqrt{\pi}
    $$

    Interdum et malesuada fames $\{1, 4, 9, \ldots\}$ ac ante ipsum primis in
    faucibus. Cras eleifend dolor eu nulla suscipit suscipit. Sed lobortis non
    felis id vulputate.

    \heading{A heading inside a block}

    Praesent consectetur mi $x^2 + y^2$ metus, nec vestibulum justo viverra
    nec. Proin eget nulla pretium, egestas magna aliquam, mollis neque. Vivamus
    dictum $\mathbf{u}^\intercal\mathbf{v}$ sagittis odio, vel porta erat
    congue sed. Maecenas ut dolor quis arcu auctor porttitor.

    \heading{Another heading inside a block}

    Sed augue erat, scelerisque a purus ultricies, placerat porttitor neque.
    Donec $P(y \mid x)$ fermentum consectetur $\nabla_x P(y \mid x)$ sapien
    sagittis egestas. Duis eget leo euismod nunc viverra imperdiet nec id
    justo.

  \end{exampleblock}

  \begin{block}{Nullam vel erat at velit convallis laoreet}

    Class aptent taciti sociosqu ad litora torquent per conubia nostra, per
    inceptos himenaeos. Phasellus libero enim, gravida sed erat sit amet,
    scelerisque congue diam. Fusce dapibus dui ut augue pulvinar iaculis.

    \begin{table}
      \centering
      \begin{tabular}{l r r c}
        \toprule
        \textbf{First column} & \textbf{Second column} & \textbf{Third column} & \textbf{Fourth} \\
        \midrule
        Foo & 13.37 & 384,394 & $\alpha$ \\
        Bar & 2.17 & 1,392 & $\beta$ \\
        Baz & 3.14 & 83,742 & $\delta$ \\
        Qux & 7.59 & 974 & $\gamma$ \\
        \bottomrule
      \end{tabular}
      \caption{A table caption.}
    \end{table}

    Donec quis posuere ligula. Nunc feugiat elit a mi malesuada consequat. Sed
    imperdiet augue ac nibh aliquet tristique. Aenean eu tortor vulputate,
    eleifend lorem in, dictum urna. Proin auctor ante in augue tincidunt
    tempor. Proin pellentesque vulputate odio, ac gravida nulla posuere
    efficitur. Aenean at velit vel dolor blandit molestie. Mauris laoreet
    commodo quam, non luctus nibh ullamcorper in. Class aptent taciti sociosqu
    ad litora torquent per conubia nostra, per inceptos himenaeos.



  \end{block}

  \begin{block}{References}

    Este documento es un ejemplo de póster académico usando LaTeX.

  \end{block}

\end{column}

\separatorcolumn
\end{columns}
\end{frame}

\end{document}
